%==============================================================================
% Sjabloon onderzoeksvoorstel bachproef
%==============================================================================
% Gebaseerd op document class `hogent-article'
% zie <https://github.com/HoGentTIN/latex-hogent-article>

% Voor een voorstel in het Engels: voeg de documentclass-optie [english] toe.
% Let op: kan enkel na toestemming van de bachelorproefcoördinator!
\documentclass{hogent-article}

% Invoegen bibliografiebestand
\addbibresource{voorstel.bib}

% Informatie over de opleiding, het vak en soort opdracht
\studyprogramme{Professionele bachelor toegepaste informatica}
\course{Bachelorproef}
\assignmenttype{Onderzoeksvoorstel}
% Voor een voorstel in het Engels, haal de volgende 3 regels uit commentaar
% \studyprogramme{Bachelor of applied information technology}
% \course{Bachelor thesis}
% \assignmenttype{Research proposal}

\academicyear{2023-2024} % TODO: pas het academiejaar aan

% TODO: Werktitel
\title{Het verbeteren van compliance- en governancevereisten bij DocShifter, een gespecialiseerd bedrijf in documentconversietechnologie, door de praktische implementatie van DevSecOps, met een focus op klantgegevensbescherming en interne softwareontwikkelingsprocessen.}

% TODO: Studentnaam en emailadres invullen
\author{Laurens De Vos}
\email{laurens.devos@student.hogent.be}
\projectrepo{https://github.com/Laurensdevos2001/bachelorproef-2023-2024-laurensDeVos.git}

% TODO: Medestudent
% Gaat het om een bachelorproef in samenwerking met een student in een andere
% opleiding? Geef dan de naam en emailadres hier
% \author{Yasmine Alaoui (naam opleiding)}
% \email{yasmine.alaoui@student.hogent.be}

% TODO: Geef de co-promotor op
\supervisor{Johnno Van De Velde}

% Binnen welke specialisatierichting uit 3TI situeert dit onderzoek zich?
% Kies uit deze lijst:
%
% - Mobile \& Enterprise development
% - AI \& Data Engineering
% - Functional \& Business Analysis
% - System \& Network Administrator
% - Mainframe Expert
% - Als het onderzoek niet past binnen een van deze domeinen specifieer je deze
%   zelf
%
\specialisation{System \& Network Administrator}
\keywords{DevSecOps, security, automation}

\begin{document}
    
    \begin{abstract}
    Dit onderzoek zal zich richten op het identificeren en aanpakken van concrete compliance- en governanceproblemen bij DocShifter, een bedrijf gespecialiseerd in documentconversietechnologie. Het onderzoekt de specifieke uitdagingen waarmee DocShifter wordt geconfronteerd met betrekking tot het beschermen van klantgegevens en het voldoen aan relevante regelgeving, zoals de Algemene Verordening Gegevensbescherming (AVG). Door middel van een uitgebreide literatuurstudie worden relevante aspecten van het probleemdomein geïdentificeerd, zoals de aard van de activiteiten van DocShifter, de specifieke compliance- en governancevereisten waaraan het bedrijf moet voldoen, en de potentiële impact van niet-naleving.
    
    Op basis van deze analyse wordt een methodologie ontwikkeld om de implementatie van DevSecOps als een hands-on verbetering voor het bedrijf te onderzoeken. De methodologie omvat een nulmeting om de huidige status van compliance en governance binnen DocShifter te beoordelen, gevolgd door de implementatie van DevSecOps-praktijken en -hulpmiddelen. Na de implementatie zal een meting worden uitgevoerd om de impact van DevSecOps op de compliance- en governancevereisten van DocShifter te evalueren.
    
    Door deze aanpak wordt niet alleen een dieper inzicht verkregen in de specifieke uitdagingen van DocShifter op het gebied van compliance en governance, maar wordt ook een praktische oplossing ontwikkeld die direct kan worden toegepast om de efficiëntie en effectiviteit van de bedrijfsprocessen te verbeteren. 
    
    \end{abstract}
    
    \tableofcontents
    
    % De hoofdtekst van het voorstel zit in een apart bestand, zodat het makkelijk
    % kan opgenomen worden in de bijlagen van de bachelorproef zelf.
    %---------- Inleiding ---------------------------------------------------------
    
    \section{Introductie}%
    \label{sec:introductie}
    In de hedendaagse digitale wereld is het voor bedrijven van groot belang om te voldoen aan strikte compliance- en governancevereisten, vooral op het gebied van klantgegevensbescherming en interne softwareontwikkelingsprocessen. DocShifter, een grote speler in documentconversietechnologie, staat voor de uitdaging om zijn \\ compliance- en governancekaders te versterken om beter te voldoen aan regelgevingen zoals de Algemene Verordening Gegevensbescherming (AVG) en andere belangrijke normen.
    
    Dit onderzoek richt zich op het identificeren en aanpakken van de uitdagingen waarmee DocShifter wordt geconfronteerd op het gebied van compliance en governance. Het belangrijkste doel is om te onderzoeken hoe de implementatie van DevSecOps kan bijdragen aan het versterken van de compliance- en governancevereisten van het bedrijf.
    
    Door een diepgaande analyse van deze uitdagingen kunnen de juiste oplossingen worden geïdentificeerd en ontwikkeld. Dit onderzoek zal gebruik maken van een grondige literatuurstudie om relevante aspecten van de probleemstelling te verkennen en een methodologie te ontwikkelen voor de implementatie van DevSecOps als een praktische verbetering voor DocShifter. Het uiteindelijke doel is om concrete oplossingen te bieden die de efficiëntie en effectiviteit van de bedrijfsprocessen zullen verbeteren.
    
    
    
    
    %---------- Stand van zaken ---------------------------------------------------
    
   \section{Literatuurstudie}%
   \label{sec:literatuurstudie}
   \subsection{Wat is DevSecOps}
   DevSecOps, een samenvoeging van development, security en operations, is een benadering die beveiliging integreert als een gedeelde verantwoordelijkheid gedurende de volledige \\ IT-levenscyclus. Volgens Red Hat (2020) gaat dit concept verder dan het traditionele DevOps-model door beveiliging te betrekken vanaf de vroegste stadia van ontwikkeling en doorlopend tot de eindfase van het project.
   Traditioneel werd beveiliging vaak gezien als een afzonderlijke taak die pas aan het einde van het ontwikkelingsproces werd uitgevoerd. Met de toegenomen behoefte aan snellere ontwikkelingscycli in DevOps, is het echter cruciaal geworden om beveiliging continu en integraal te maken in het gehele proces. Dit betekent dat beveiligingsteam vanaf het begin betrokken zijn bij het project, en dat beveiligingscontroles worden geautomatiseerd en geschikte tools worden geselecteerd.
   De nadruk van DevSecOps ligt op het belang van vroegtijdige en doorlopende beveiliging. Dit benadrukt niet alleen het belang van het automatiseren van beveiligingscontroles, maar ook de noodzaak om beveiligingsteams vanaf het begin van het ontwikkelingsproces te betrekken.
   Effectieve implementatie van DevSecOps vereist dat beveiliging wordt geïntegreerd gedurende de volledige levenscyclus van applicaties. Automatisering speelt een cruciale rol in het vereenvoudigen van repetitieve taken en het handhaven van de ontwikkelingssnelheid. Bovendien breidt DevSecOps zich uit naar moderne technologieën zoals containers en microservices, die aangepaste beveiligingspraktijken vereisen om \\ applicatie- en infrastructuurbeveiliging te waarborgen \autocite{redhat2023}.
   
   \subsection{Compliance in DevSecOps}
   
   Het integreren van compliance in DevSecOps is essentieel om te voldoen aan industrienormen, wetten en regelgeving. Volgens Zeeshan (2020) moeten organisaties ervoor zorgen dat hun processen niet alleen veilig zijn, maar ook voldoen aan regelgeving zoals de GDPR (voor Europese landen), CCPA, FISMA en andere relevante standaarden. Compliance moet een continu proces zijn dat is ingebed in de gehele DevSecOps-pijplijn, met geïntegreerde controles gedurende de\\ontwikkelings- en operationele fasen.
   Automatisering speelt een cruciale rol bij het handhaven van compliance. Door geautomatiseerde tools te gebruiken die continu de naleving van regelgeving controleren, kunnen organisaties ervoor zorgen dat compliance-eisen consistent worden nageleefd zonder de snelheid van de ontwikkeling te vertragen. "Compliance-as-code" is een benadering waarbij compliance-eisen worden geprogrammeerd in dezelfde taal en tools die ontwikkelaars gebruiken voor applicatieontwikkeling, waardoor nalevingsregels automatisch worden toegepast en gevalideerd tijdens het bouw- en implementatieproces.
   Om compliance effectief te handhaven, moeten ontwikkelaars en operationele teams regelmatig worden getraind in de relevante wetten en regelgeving. Tools en platforms voor het beheren van compliance kunnen integreren met bestaande CI/CD-pijplijnen, waardoor elke codewijziging automatisch wordt gecontroleerd op naleving van de relevante regelgeving. Dit omvat statische en dynamische beveiligingstests en het bewaken van de integriteit van softwarecomponenten.
   Continue monitoring en rapportage zijn essentieel om naleving te waarborgen. Monitoringtools die real-time inzicht geven in de nalevingsstatus, stellen organisaties in staat snel te reageren op mogelijke schendingen en corrigerende maatregelen te nemen. Dit versterkt het vertrouwen van klanten en partners in de beveiligingspraktijken van de organisatie.
   Door compliance effectief te integreren in het DevSecOps-proces, kunnen organisaties niet alleen voldoen aan de noodzakelijke regelgeving, maar ook hun algehele beveiligingshouding verbeteren en de risico's van datalekken en andere beveiligingsincidenten verminderen. Dit zorgt voor een robuust en betrouwbaar ontwikkelings- en operationeel proces en versterkt het vertrouwen van klanten en partners.\autocite{Zeeshan2020}
   
   Sinds 2020 is er een groeiende interesse in het bespreken van compliance-aspecten binnen DevSecOps, zoals blijkt uit een toename van publicaties op dit gebied. Volgens Ramaj (2022) komt dit door strengere regelgeving en toenemende cybersecuritybedreigingen, waardoor het belang van compliance steeds duidelijker wordt.
   Compliance initiatie omvat het definiëren van de gewenste staat van naleving van beveiligingsmaatregelen en het specificeren van nalevingsvereisten. Compliance management houdt alle processen in die verband houden met het beheer van naleving, zoals automatisering, testen en verificatie, validatie, controle en monitoring, bewustwording en training, en evaluatie. Technische compliance-aspecten betreffen de technische middelen die worden gebruikt voor het beheren van naleving, zoals compliance-as-code en diverse nalevingstools.
   Organisaties moeten investeren in geautomatiseerde nalevingscontroles, training van personeel en het gebruik van geschikte tools en technologieën om effectieve compliance te waarborgen in een DevSecOps-omgeving. Door compliance te integreren in het ontwikkelingsproces kunnen organisaties beter voldoen aan regelgeving, risico's verminderen en vertrouwen opbouwen bij klanten en partners.\autocite{Ramaj2022}
   
   \subsection{Beveiligingsrisico's in DevSecOps}
   
    Volgens Chandramouli (2022) vereist het integreren van beveiliging in de softwareontwikkelingslevenscyclus (SDLC), vooral met de complexiteit van microservices en cloudimplementaties, het aannemen van een DevSecOps-model. Dit model zorgt ervoor dat beveiliging een integraal onderdeel is van het ontwikkelingsproces in plaats van een bijzaak. Door het implementeren van geautomatiseerde beveiligingstests, robuuste toegangscontroles en uitgebreide tools voor kwetsbaarheidsbeheer, kunnen organisaties het risico op aanvallen die gebruikmaken van de ingewikkeldheden van moderne softwareomgevingen aanzienlijk verminderen. Door beveiliging als prioriteit te stellen gedurende het gehele ontwikkelingsproces, kunnen teams software ontwikkelen die zowel veerkrachtig als wendbaar is.\\ \autocite{Chandramouli2022}
   
   \subsection{Best Practices voor het Implementeren van DevSecOps}
   
   Volgens Rajavi en Desai (2021) vereist het succesvol implementeren van DevSecOps een aantal best practices. Ten eerste is training cruciaal. Goed ontworpen trainingsprogramma's kunnen het bewustzijn van beveiliging vergroten en benadrukken hoe individuen kunnen bijdragen aan beveiliging. Ten tweede is het implementeren van beveiligingstools van groot belang. Deze tools moeten geïntegreerd zijn in bestaande systemen en continue monitoring en oplossingen bieden tijdens het ontwikkelingsproces. Het is daarom dus ook het belangrijkste onderdeel om aan DevSecOps te doen. Ten derde is het nemen van proactieve maatregelen essentieel. Dit omvat het beveiligen van alle componenten zoals code, databases en netwerken, door middel van diverse controles, veilig coderen en het filteren van verkeer. \autocite{RajaviDesai2021}
    
    %---------- Methodologie ------------------------------------------------------
    \section{Methodologie}%
    \label{sec:methodologie}
    
    Om de implementatie van DevSecOps te onderzoeken en te begrijpen hoe deze kan bijdragen aan het versterken van compliance- en governancevereisten, zal er een methodologie voorgesteld worden die verschillende fasen omvat.
    Om te beginnen zal de eerste fase een grondige analyse omvatten van de huidige compliance- en governancevereisten van DocShifter. Het doel van deze eerste fase is specifieke uitdagingen en knelpunten te identificeren en duidelijke doelstellingen voor de implementatie van DevSecOps vast te stellen.
    in een volgende fase zal er een literatuurstudie worden uitgevoerd. er zal gekeken worden naar relevante literatuur over DevSecOps, compliance-frameworks zoals GDPR, en governanceprincipes. Hierbij zal speciale aandacht worden besteed aan best practices en case studies die relevant zijn voor de context van DocShifter..
    Vervolgens zullen de Resultaten en Conclusies worden samengevat, waarbij de verworven informatie wordt bekeken in het kader van de onderzoeksvraag en doelstellingen. Er zullen Aanbevelingen worden geformuleerd voor eventuele toekomstige implementaties van DevSecOps binnen DocShifter.
    Tot slot zal de verworven informatie van vorige fasen worden gebruikt om zorgvuldig één of meerdere tools te kiezen en uit te werken zodat Docshifter aan zijn compliance- en governance noden zal kunnen voldoen. 

    
    %---------- Verwachte resultaten ----------------------------------------------
    \section{Verwacht resultaat, conclusie}%
    \label{sec:verwachte_resultaten}
    Op basis van de voorgestelde methodologie zijn verschillende resultaten te verwachten. Allereerst wordt verwacht dat een diepgaand inzicht wordt verkregen in de huidige compliance- en governancevereisten van DocShifter, inclusief identificatie van specifieke uitdagingen en knelpunten. Dit inzicht zal dienen als een solide basis om potentiële problemen aan te pakken en effectieve oplossingen te formuleren.
    Verder wordt verwacht dat er een helder beeld ontstaat van hoe DevSecOps kan worden geïmplementeerd om deze potentiële problemen aan te pakken. Hierbij zullen gedetailleerde aanbevelingen worden geformuleerd, die als leidraad zullen dienen voor toekomstige stappen in de implementatie van DevSecOps binnen DocShifter. Dit zal niet alleen helpen om de compliance- en governancevereisten te versterken, maar ook om de algehele beveiligingspositie van het bedrijf te verbeteren.
    Daarnaast zal de literatuurstudie resulteren in de selectie en uitwerking van één of meerdere tools die specifiek zijn gericht op het verminderen van governance- en complianceproblemen binnen DocShifter. Na de implementatie van deze tools wordt verwacht dat een aanzienlijk deel van de governance- en compliancekwesties effectief zal worden aangepakt, waardoor DocShifter beter in staat zal zijn te voldoen aan relevante regelgeving en industriestandaarden.
    In conclusie zal dit onderzoek niet alleen bijdragen aan een beter begrip van de compliance- en governancevereisten van DocShifter, maar zal het ook concrete oplossingen bieden om deze te versterken. De resulterende aanbevelingen en implementaties zullen een significante impact hebben op de beveiligingspraktijken en het algehele operationele proces van het bedrijf, wat zal leiden tot een verbeterde efficiëntie, naleving en vertrouwen van klanten en partners.
         
    
    \printbibliography[heading=bibintoc]
    
\end{document}