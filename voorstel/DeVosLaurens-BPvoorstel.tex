%==============================================================================
% Sjabloon onderzoeksvoorstel bachproef
%==============================================================================
% Gebaseerd op document class `hogent-article'
% zie <https://github.com/HoGentTIN/latex-hogent-article>

% Voor een voorstel in het Engels: voeg de documentclass-optie [english] toe.
% Let op: kan enkel na toestemming van de bachelorproefcoördinator!
\documentclass{hogent-article}

% Invoegen bibliografiebestand
\addbibresource{voorstel.bib}

% Informatie over de opleiding, het vak en soort opdracht
\studyprogramme{Professionele bachelor toegepaste informatica}
\course{Bachelorproef}
\assignmenttype{Onderzoeksvoorstel}
% Voor een voorstel in het Engels, haal de volgende 3 regels uit commentaar
% \studyprogramme{Bachelor of applied information technology}
% \course{Bachelor thesis}
% \assignmenttype{Research proposal}

\academicyear{2023-2024} % TODO: pas het academiejaar aan

% TODO: Werktitel
\title{Het verbeteren van compliance- en governancevereisten bij DocShifter, een gespecialiseerd bedrijf in documentconversietechnologie, door de praktische implementatie van DevSecOps, met een focus op klantgegevensbescherming en interne softwareontwikkelingsprocessen.}

% TODO: Studentnaam en emailadres invullen
\author{Laurens De Vos}
\email{laurens.devos@student.hogent.be}
\projectrepo{https://github.com/Laurensdevos2001/bachelorproef-2023-2024-laurensDeVos.git}

% TODO: Medestudent
% Gaat het om een bachelorproef in samenwerking met een student in een andere
% opleiding? Geef dan de naam en emailadres hier
% \author{Yasmine Alaoui (naam opleiding)}
% \email{yasmine.alaoui@student.hogent.be}

% TODO: Geef de co-promotor op
\supervisor{Johnno Van De Velde}

% Binnen welke specialisatierichting uit 3TI situeert dit onderzoek zich?
% Kies uit deze lijst:
%
% - Mobile \& Enterprise development
% - AI \& Data Engineering
% - Functional \& Business Analysis
% - System \& Network Administrator
% - Mainframe Expert
% - Als het onderzoek niet past binnen een van deze domeinen specifieer je deze
%   zelf
%
\specialisation{System \& Network Administrator}
\keywords{DevSecOps, security, automation}

\begin{document}
    
    \begin{abstract}
    
    DocShifter, een bedrijf gespecialiseerd in documentconversietechnologie, kampt met significante uitdagingen op het gebied van compliance en governance, vooral met betrekking tot de bescherming van klantgegevens en naleving van de Algemene Verordening Gegevensbescherming (AVG). Deze problemen bedreigen niet alleen de bedrijfsvoering, maar ook de reputatie van DocShifter, wat de urgentie van dit onderzoek benadrukt.
    
    Om deze uitdagingen aan te pakken, is een uitgebreide literatuurstudie uitgevoerd. Deze studie analyseert de specifieke activiteiten van DocShifter, de compliance- en governancevereisten waaraan het moet voldoen, en de mogelijke gevolgen van niet-naleving. Op basis van deze analyse is een methodologie ontwikkeld, waarin een nulmeting de huidige status van compliance en governance evalueert. Vervolgens worden DevSecOps-praktijken geïmplementeerd, waarna de impact op de naleving van regelgeving en gegevensbescherming wordt gemeten.
    
    De resultaten van dit onderzoek bieden waardevolle inzichten in de compliance- en governance-uitdagingen van DocShifter. Bovendien wordt een praktische oplossing voorgesteld die de efficiëntie en effectiviteit van de bedrijfsprocessen verbetert, wat bijdraagt aan een versterkte naleving van regelgeving en betere gegevensbescherming.
    
    
    \end{abstract}
    
    \tableofcontents
    
    % De hoofdtekst van het voorstel zit in een apart bestand, zodat het makkelijk
    % kan opgenomen worden in de bijlagen van de bachelorproef zelf.
    %---------- Inleiding ---------------------------------------------------------
    
    \section{Introductie}%
    \label{sec:introductie}

    In de hedendaagse digitale wereld is het voor bedrijven cruciaal om te voldoen aan strikte compliance- en governancevereisten, met name op het gebied van klantgegevensbescherming en interne softwareontwikkelingsprocessen. DocShifter, een vooraanstaande speler in documentconversietechnologie, wordt momenteel geconfronteerd met specifieke uitdagingen bij het naleven van de Algemene Verordening Gegevensbescherming (AVG). Deze uitdagingen hebben betrekking op het ontoereikend versleutelen van klantgegevens of het niet tijdig en snel genoeg updaten van software om aan de vereisten te voldoen, wat de organisatie blootstelt aan aanzienlijke risico's op legaal vlak.
    
    \noindent De huidige aanpak van DocShifter blijkt ontoereikend te zijn, omdat veiligheidsmaatregelen over het hoofd kunnen gezien worden door een minder goede aanpak van software uit te brengen. Dit roept de vraag op hoe DocShifter zijn compliance- en governancekaders kan versterken om beter te voldoen aan de AVG en andere relevante normen.
    
    \noindent De centrale onderzoeksvraag in dit onderzoek luidt dan ook: Hoe kan de implementatie van DevSecOps bijdragen aan het verbeteren van de\\ compliance- en governancevereisten van DocShifter, specifiek in het kader van de AVG? Deze vraag wordt opgesplitst in de volgende deelvragen:
    
    \begin{itemize}
        \item  Wat zijn de huidige tekortkomingen in de compliance- en governanceaanpak van DocShifter ten aanzien van de AVG? Welke risico's lopen zij door deze tekortkomingen?
    \end{itemize}
    
    \begin{itemize}
        \item  Hoe kunnen DevSecOps-praktijken specifiek bijdragen aan het verbeteren van de naleving van de AVG bij DocShifter? Welke methoden en tools zijn het meest geschikt voor deze implementatie?
    \end{itemize}
    
    \noindent Op basis van deze deelvragen worden de volgende doelstellingen geformuleerd:
    
    \begin{itemize}
        \item Het in kaart brengen van de huidige \\ compliance- en governance-uitdagingen bij DocShifter met betrekking tot de AVG.
    \end{itemize}
    
    \begin{itemize}
        \item Het ontwikkelen en voorstellen van een \\ DevSecOps-strategie die specifiek gericht is op het aanpakken van deze uitdagingen.
    \end{itemize}
    \begin{itemize}
        \item Het evalueren van de effectiviteit van deze strategie in termen van verbeterde compliance en gegevensbescherming.
    \end{itemize}
    
    \noindent Door een grondige analyse van deze uitdagingen en mogelijke oplossingen, streeft dit onderzoek ernaar om concrete aanbevelingen te doen die de efficiëntie en effectiviteit van de bedrijfsprocessen bij DocShifter zullen verbeteren.
    
    
    
    
    %---------- Stand van zaken ---------------------------------------------------
    
   \section{Literatuurstudie}%
   \label{sec:literatuurstudie}
   \subsection{Wat is DevSecOps}
   DevSecOps, een samenvoeging van development, security en operations, is een benadering die beveiliging integreert als een gedeelde verantwoordelijkheid gedurende de volledige \\ IT-levenscyclus. Volgens Red Hat (2020) gaat dit concept verder dan het traditionele DevOps-model door beveiliging te betrekken vanaf de vroegste stadia van ontwikkeling en doorlopend tot de eindfase van het project.
   Traditioneel werd beveiliging vaak gezien als een afzonderlijke taak die pas aan het einde van het ontwikkelingsproces werd uitgevoerd. Met de toegenomen behoefte aan snellere ontwikkelingscycli in DevOps, is het echter cruciaal geworden om beveiliging continu en integraal te maken in het gehele proces. Dit betekent dat beveiligingsteam vanaf het begin betrokken zijn bij het project, en dat beveiligingscontroles worden geautomatiseerd en geschikte tools worden geselecteerd.
   De nadruk van DevSecOps ligt op het belang van vroegtijdige en doorlopende beveiliging. Dit benadrukt niet alleen het belang van het automatiseren van beveiligingscontroles, maar ook de noodzaak om beveiligingsteams vanaf het begin van het ontwikkelingsproces te betrekken.
   Effectieve implementatie van DevSecOps vereist dat beveiliging wordt geïntegreerd gedurende de volledige levenscyclus van applicaties. Automatisering speelt een cruciale rol in het vereenvoudigen van repetitieve taken en het handhaven van de ontwikkelingssnelheid. Bovendien breidt DevSecOps zich uit naar moderne technologieën zoals containers en microservices, die aangepaste beveiligingspraktijken vereisen om \\ applicatie- en infrastructuurbeveiliging te waarborgen \autocite{redhat2023}.
   
   \subsection{Compliance in DevSecOps}
   
   Het integreren van compliance in DevSecOps is essentieel om te voldoen aan industrienormen, wetten en regelgeving. Volgens Zeeshan (2020) moeten organisaties ervoor zorgen dat hun processen niet alleen veilig zijn, maar ook voldoen aan regelgeving zoals de GDPR (voor Europese landen), CCPA, FISMA en andere relevante standaarden. Compliance moet een continu proces zijn dat is ingebed in de gehele DevSecOps-pijplijn, met geïntegreerde controles gedurende de\\ontwikkelings- en operationele fasen.
   Automatisering speelt een cruciale rol bij het handhaven van compliance. Door geautomatiseerde tools te gebruiken die continu de naleving van regelgeving controleren, kunnen organisaties ervoor zorgen dat compliance-eisen consistent worden nageleefd zonder de snelheid van de ontwikkeling te vertragen. "Compliance-as-code" is een benadering waarbij compliance-eisen worden geprogrammeerd in dezelfde taal en tools die ontwikkelaars gebruiken voor applicatieontwikkeling, waardoor nalevingsregels automatisch worden toegepast en gevalideerd tijdens het bouw- en implementatieproces.
   Om compliance effectief te handhaven, moeten ontwikkelaars en operationele teams regelmatig worden getraind in de relevante wetten en regelgeving. Tools en platforms voor het beheren van compliance kunnen integreren met bestaande CI/CD-pijplijnen, waardoor elke codewijziging automatisch wordt gecontroleerd op naleving van de relevante regelgeving. Dit omvat statische en dynamische beveiligingstests en het bewaken van de integriteit van softwarecomponenten.
   Continue monitoring en rapportage zijn essentieel om naleving te waarborgen. Monitoringtools die real-time inzicht geven in de nalevingsstatus, stellen organisaties in staat snel te reageren op mogelijke schendingen en corrigerende maatregelen te nemen. Dit versterkt het vertrouwen van klanten en partners in de beveiligingspraktijken van de organisatie.
   Door compliance effectief te integreren in het DevSecOps-proces, kunnen organisaties niet alleen voldoen aan de noodzakelijke regelgeving, maar ook hun algehele beveiligingshouding verbeteren en de risico's van datalekken en andere beveiligingsincidenten verminderen. Dit zorgt voor een robuust en betrouwbaar ontwikkelings- en operationeel proces en versterkt het vertrouwen van klanten en partners.\autocite{Zeeshan2020}\\
   
   Sinds 2020 is er een groeiende interesse in het bespreken van compliance-aspecten binnen DevSecOps, zoals blijkt uit een toename van publicaties op dit gebied. Volgens Ramaj (2022) komt dit door strengere regelgeving en toenemende cybersecuritybedreigingen, waardoor het belang van compliance steeds duidelijker wordt.
   Compliance initiatie omvat het definiëren van de gewenste staat van naleving van beveiligingsmaatregelen en het specificeren van nalevingsvereisten. Compliance management houdt alle processen in die verband houden met het beheer van naleving, zoals automatisering, testen en verificatie, validatie, controle en monitoring, bewustwording en training, en evaluatie. Technische compliance-aspecten betreffen de technische middelen die worden gebruikt voor het beheren van naleving, zoals compliance-as-code en diverse nalevingstools.
   Organisaties moeten investeren in geautomatiseerde nalevingscontroles, training van personeel en het gebruik van geschikte tools en technologieën om effectieve compliance te waarborgen in een DevSecOps-omgeving. Door compliance te integreren in het ontwikkelingsproces kunnen organisaties beter voldoen aan regelgeving, risico's verminderen en vertrouwen opbouwen bij klanten en partners.\autocite{Ramaj2022}
   
   \subsection{Beveiligingsrisico's in DevSecOps}
   
    Volgens Chandramouli (2022) vereist het integreren van beveiliging in de softwareontwikkelingslevenscyclus (SDLC), vooral met de complexiteit van microservices en cloudimplementaties, het aannemen van een DevSecOps-model. Dit model zorgt ervoor dat beveiliging een integraal onderdeel is van het ontwikkelingsproces in plaats van een bijzaak. Door het implementeren van geautomatiseerde beveiligingstests, robuuste toegangscontroles en uitgebreide tools voor kwetsbaarheidsbeheer, kunnen organisaties het risico op aanvallen die gebruikmaken van de ingewikkeldheden van moderne softwareomgevingen aanzienlijk verminderen. Door beveiliging als prioriteit te stellen gedurende het gehele ontwikkelingsproces, kunnen teams software ontwikkelen die zowel veerkrachtig als wendbaar is.\\ \autocite{Chandramouli2022}
   
   \subsection{Best Practices voor het Implementeren van DevSecOps}
   
   Volgens Rajavi en Desai (2021) vereist het succesvol implementeren van DevSecOps een aantal best practices. Ten eerste is training cruciaal. Goed ontworpen trainingsprogramma's kunnen het bewustzijn van beveiliging vergroten en benadrukken hoe individuen kunnen bijdragen aan beveiliging. Ten tweede is het implementeren van beveiligingstools van groot belang. Deze tools moeten geïntegreerd zijn in bestaande systemen en continue monitoring en oplossingen bieden tijdens het ontwikkelingsproces. Het is daarom dus ook het belangrijkste onderdeel om aan DevSecOps te doen. Ten derde is het nemen van proactieve maatregelen essentieel. Dit omvat het beveiligen van alle componenten zoals code, databases en netwerken, door middel van diverse controles, veilig coderen en het filteren van verkeer. \autocite{RajaviDesai2021}
    
    %---------- Methodologie ------------------------------------------------------
    \section{Methodologie}%
    \label{sec:methodologie}
    
    \noindent Om de implementatie van DevSecOps te onderzoeken en te begrijpen hoe deze kan bijdragen aan het versterken van compliance- en governancevereisten, zal er een methodologie voorgesteld worden die verschillende fasen omvat. 
    
    \subsection{Fase 1: Analyse van huidige compliance- en governancevereisten}
    \noindent De eerste fase omvat een grondige analyse van de huidige compliance- en governancevereisten van DocShifter. Dit proces zal inhouden:
    
    \begin{itemize}
        \item Stakeholder interviews: Relevante stakeholders zoals ontwikkelaars en devops engineers interviewen om inzicht te krijgen in huidige vereisten en knelpunten binnen DocShifter. Dit zal helpen bij het verstrekken van belangrijke informatie over de behoeften en verwachtingen.
        \item Procesanalyse: In de interviews zal ook bevraagd worden naar de huidige processen en workflows binnen DocShifter om inefficiënties en risico's te identificeren.
        \item requirement specificatie: Het opstellen van een gedetailleerd requirementanalyse document waarin de vereisten voor de implementatie van DevSecOps worden vastgelegd. Dit zal dienen als basis voor de verdere ontwikkeling.
    \end{itemize}
    
    \subsection{Fase 2: Literatuurstudie}
    \noindent Een uitgebreide literatuurstudie zal worden uitgevoerd om inzicht te krijgen in de huidige stand van zaken rondom DevSecOps, compliance-frameworks (zoals GDPR), en governanceprincipes. Hierbij wordt speciale aandacht besteed aan:
    
    \begin{itemize}
        \item Best practices en case studies die relevant zijn voor de context van DocShifter.
        \item Mogelijke benaderingen en tools die kunnen worden toegepast om de gestelde doelstellingen te bereiken.
    \end{itemize}
    
    \subsection{Fase 3: Opstellen van een long list}
    \noindent Op basis van de literatuurstudie en de analyse van de huidige situatie zal een long list van mogelijke DevSecOps-implementaties worden opgesteld. Deze lijst omvat frameworks, tools en best practices die in aanmerking komen voor verdere evaluatie. Elk item op deze lijst wordt beoordeeld op:
    
    \begin{itemize}
        \item Relevantie voor DocShifter.
        \item Toepasbaarheid in de huidige omgeving
        \item potentieel om de gestelde doelstellingen te bereiken.
    \end{itemize}

    \subsection{Fase 4: Evaluatie en Short List}
    \noindent De long list wordt geëvalueerd om een short list van de meest geschikte opties te bepalen. Het evaluatieproces omvat criteria zoals:
    
    \begin{itemize}
        \item Effectiviteit
        \item implementatiegemak
        \item Schaalbaarheid
        \item ondersteuning voor compliance- en governancevereisten.
    \end{itemize}
    
    \subsection{Fase 5: Proof of Concept (PoC)}
    \noindent De meest veelbelovende DevSecOps-oplossingen worden geselecteerd voor een Proof of Concept (PoC). De PoC omvat:
    
    \begin{itemize}
        \item Praktische implementaties en tests van de gekozen oplossingen.
        \item Het vastleggen van vooraf bepaalde, objectieve succescriteria om de effectiviteit van de oplossingen te evalueren. Deze criteria kunnen prestatie-indicatoren, compliance met regelgeving, en verbeteringen in processen omvatten.
        \item Een gedetailleerde analyse en documentatie van de resultaten van de PoC om inzicht te geven in de prestaties en voordelen van de geïmplementeerde oplossingen.
    \end{itemize}

    \subsection{Fase 6: Resultaten en Aanbevelingen}
    \noindent Tot slot worden de resultaten en conclusies van het gehele onderzoek samengevat. Hierbij wordt de verworven informatie beoordeeld in functie van de onderzoeksvraag en doelstellingen. Op basis hiervan worden aanbevelingen geformuleerd voor:
    
    \begin{itemize}
        \item Eventuele toekomstige implementaties van DevSecOps binnen DocShifter.
        \item Het optimaliseren van de governance- en compliancevereisten.
        \item Het waarborgen van een continue verbetering van de DevSecOps-processen binnen het bedrijf.
    \end{itemize}
    
    %---------- Verwachte resultaten ----------------------------------------------
    \section{Verwacht resultaat, conclusie}%
    \label{sec:verwachte_resultaten}
    Het onderzoek zal naar verwachting resulteren in een diepgaand inzicht in de huidige compliance- en governance-uitdagingen van DocShifter. Door middel van een systematische analyse en literatuurstudie wordt verwacht dat er een duidelijke strategie zal worden ontwikkeld voor de implementatie van DevSecOps binnen het bedrijf. Deze strategie resulteert in een selectie van relevante tools en best practices die specifiek zijn afgestemd op de behoeften van DocShifter. Bovendien zal de Proof of Concept een praktisch bewijs leveren van de effectiviteit van deze aanpak, wat zal bijdragen aan het verbeteren van interne processen.
    
    In conclusie zal dit onderzoek waardevolle inzichten bieden in hoe DocShifter zijn compliance- en governancevereisten kan versterken door de implementatie van DevSecOps. De voorgestelde oplossingen zullen niet alleen de huidige uitdagingen aanpakken, maar ook een duurzame basis leggen voor toekomstige verbeteringen in de beveiliging en efficiëntie binnnen DocShifter. De resultaten van dit onderzoek zullen DocShifter in staat stellen om beter te voldoen aan relevante regelgeving, wat uiteindelijk zal bijdragen aan een versterkt vertrouwen van klanten en stakeholders.
    
    
         
    
    \printbibliography[heading=bibintoc]
    
\end{document}